\chapter{Zusammenfassung und Ausblick} \label{chap:Zusammenfassung}
\thispagestyle{empty}

Nachdem in der Einleitung die Einordnung und Vorstellung des Themas erfolgt wurde im Grundlagenkapitel auf die zu testende Architektur eingegangen und eine theoretische Hinführung zum Thema des szenariobasierten Testen gegeben und die Relevanz von Softwaretest über den gesamten Produktlebenszyklus verdeutlicht.

Nach einer Anforderungsdefinition wurden die Möglichkeiten, die des MATLAB\textsuperscript{\textregistered} Unit Test Framework für die Umsetzung der Aufgabe bietet, ergründet und daraus ein geeignetes Konzept zur Implementierung erarbeitet. Es ermöglicht die einfache Parametrierung von Szenarien und abtesten von KPIs. Neue Szenarien können durch das Anlegen einer neuen Testklasse, Definition von KPIs und Paramertern hinzugefügt werden und müssen im jeweiligen Job für ein Fahrzeug ergänzt werden. Sollten in zukünftigen Entwicklungen weitere Fahrzeuge hinzukommen ist lediglich ein neuer Job in der .yaml-Datei für die GitLab Pipeline anzulegen. 

Nach Analyse der Simulationsumgebung und einer Literaturrecherche wurden einige Szenarien und die dazugehörigen KPIs implementiert. Um aus den ermittelten funktionalen Szenarien konkrete Szenarien zu generieren wurde das Latin Hypercube Sampling verwendet. Eine zukünftige Verbesserung dieser Sampling-Methode hinsichtlich der Generierung von kritischeren Parameterkombinationen ist denkbar. Derzeit wird, z. B. bei der Kurvenradien, aquidistant über den gesamten Parameterbereich gesampelt. engere Kurven sind allerdings deutlich kritischer als weitere Kurven, weswegen eine engere Abtastung bei kleineren Radien sinnvoller ist. Eine Möglichkeit der Umsetzung dieser Anforderung ist die Äquivalenzklassenbildung. Dabei werden für jeden Parameter Bereiche gesucht, in denen das erwartete Systemverhalten gleich ist. Es wird nun so gesampelt, dass entweder aus jeder Äquivalenzklasse genau ein Wert gebildet wird, analog zum LHS, oder aber die Anzahl an der Samples in jeder Aquivalenzklasse gleich groß ist.

Die automatisierte Ausführung der Testskripte und Testklassen in einer CI Pipeline wurde erreicht. Die Testergebnisse werden übersichtlich in ein PDF-Datei verpackt oder können über den Browser im GitLab angezeigt werden. Die Testberichte geben dem Entwickler einen guten Überblick, welche Szenarien fehlgeschlagen sind und warum.