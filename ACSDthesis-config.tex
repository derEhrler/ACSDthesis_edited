%%%%%%%%%%%%%%%%%%%%%%%%%%%%%%%%%%%%%%%%%
% Classicthesis Typographic Thesis
% Configuration File
%
% Important note:
% The main lines to change in this file are in the DOCUMENT VARIABLES
% section, the rest of the file is for advanced configuration.
%
%%%%%%%%%%%%%%%%%%%%%%%%%%%%%%%%%%%%%%%%%
%----------------------------------------------------------------------------------------
%	CHARACTER ENCODING
%----------------------------------------------------------------------------------------

\PassOptionsToPackage{utf8}{inputenc} % Set the encoding of your files. UTF-8 is the only sensible encoding nowadays. If you can't read äöüßáéçèê∂åëæƒÏ€ then change the encoding setting in your editor, not the line below. If your editor does not support utf8 use another editor!
\usepackage{inputenc}

\PassOptionsToPackage{listings, floatperchapter}{ACSDthesis}
% Available options: drafting parts manychapters floatperchapter listings

%----------------------------------------------------------------------------------------
%	USEFUL COMMANDS
%----------------------------------------------------------------------------------------

\newcommand{\ie}{i.\,e.}
\newcommand{\Ie}{I.\,e.}
\newcommand{\eg}{e.\,g.}
\newcommand{\Eg}{E.\,g.} 

\newcounter{dummy} % Necessary for correct hyperlinks (to index, bib, etc.)
\providecommand{\mLyX}{L\kern-.1667em\lower.25em\hbox{Y}\kern-.125emX\@}
\newlength{\abcd} % for ab..z string length calculation

%----------------------------------------------------------------------------------------
%	BIBLIOGRAPHY SETUP
%----------------------------------------------------------------------------------------

\usepackage{csquotes}
\PassOptionsToPackage{%
%backend=bibtex8, % Instead of bibtex
backend=biber,bibencoding=ascii,%
language=auto,%
style=numeric-comp,%
%%style=authoryear-comp, % Author 1999, 2010
%bibstyle=authoryear,dashed=false, % dashed: substitute rep. author with ---
sorting=nyt, % name, year, title
maxbibnames=10, % default: 3, et al.
%backref=true,%
natbib=true % natbib compatibility mode (\citep and \citet still work)
}{biblatex}
\usepackage{biblatex}

\DefineBibliographyStrings{english}{%
  mathesis = {Master's thesis},
}

\DefineBibliographyStrings{german}{%
  mathesis = {Masterarbeit},
  phdthesis = {Dissertation},
}

%----------------------------------------------------------------------------------------
%	PACKAGES
%----------------------------------------------------------------------------------------
 
%\PassOptionsToPackage{fleqn}{amsmath} % equations flushed left
\usepackage{amsmath, amsthm, amssymb, amsfonts} 
\DeclareMathOperator{\argmin}{argmin}

\PassOptionsToPackage{T1}{fontenc} % T2A for cyrillics
\usepackage{fontenc}
\usepackage{textcomp} % Fix warning with missing font shapes
\usepackage{scrhack} % Fix warnings when using KOMA with listings package  
\usepackage{xspace} % To get the spacing after macros right
\usepackage{mparhack} % To get marginpar right
\usepackage{fixltx2e} % Fixes some LaTeX stuff 
\PassOptionsToPackage{printonlyused, smaller}{acronym} % Include printonlyused in the first bracket to only show acronyms used in the text
\usepackage{acronym} % Nice macros for handling all acronyms in the thesis

\PassOptionsToPackage{pdftex}{graphicx}
\usepackage{graphicx} 
\usepackage{epsfig, epstopdf, tikz, calc}
\usepackage{enumerate, units, here}
\usepackage{multicol} % xcolor already loaded in style file
%\usepackage{todonotes}% um sich notizen zu machen wie z.b. dummie Bilder
%\usepackage{catchfile}
%\usepackage{autobreak} %kann in align umgebung seitenumbruch autom.

%----------------------------------------------------------------------------------------
%	FLOATS: TABLES, FIGURES AND CAPTIONS SETUP
%----------------------------------------------------------------------------------------

\usepackage{tabularx} % Better tables
\setlength{\extrarowheight}{3pt} % Increase table row height
\newcommand{\tableheadline}[1]{\multicolumn{1}{c}{\spacedlowsmallcaps{#1}}}
\newcommand{\myfloatalign}{\centering} % To be used with each float for alignment
\usepackage{caption}
\captionsetup{font=small, labelfont=bf}
\usepackage{subcaption}  

%----------------------------------------------------------------------------------------
%	CODE LISTINGS AND ALGOITHMS SETUP
%----------------------------------------------------------------------------------------

\usepackage{listings} 
%\lstset{emph={trueIndex,root},emphstyle=\color{BlueViolet}}%\underbar} % For special keywords
\lstset{language=[LaTeX]Tex,%C++ % Specify the language(s) for listings here
morekeywords={PassOptionsToPackage,selectlanguage},
keywordstyle=\color{RoyalBlue}, % Add \bfseries for bold
basicstyle=\small\ttfamily, % Makes listings a smaller font size and a different font
%identifierstyle=\color{NavyBlue}, % Color of text inside brackets
commentstyle=\color{BrickRed}\ttfamily, % Color of comments
stringstyle=\rmfamily, % Font type to use for strings
numbers=left, % Change left to none to remove line numbers
numberstyle=\scriptsize, % Font size of the line numbers
stepnumber=5, % Increment of line numbers
numbersep=8pt, % Distance of line numbers from code listing
showstringspaces=false, % Sets whether spaces in strings should appear underlined
breaklines=true, % Force the code to stay in the confines of the listing box
%frameround=ftff, % Uncomment for rounded frame
%frame=single, % Frame border - none/leftline/topline/bottomline/lines/single/shadowbox/L
belowcaptionskip=.75\baselineskip % Space after the "Listing #: Desciption" text and the listing box
}

\PassOptionsToPackage{linesnumbered, ruled}{algorithm2e}
\usepackage{algorithm2e}

%----------------------------------------------------------------------------------------

\usepackage{ACSDthesis} 

%----------------------------------------------------------------------------------------
%	USING DIFFERENT FONTS
%----------------------------------------------------------------------------------------

\usepackage[sfdefault]{roboto}
\usepackage{arev}

\newcommand{\bs}{\boldsymbol}
\newcommand{\bm}{\boldsymbol}

%\DeclareMathSizes{12}{11}{8.5}{7}
\linespread{1.1}
\setcapindent{1em}

%----------------------------------------------------------------------------------------
% RENEW DOTS
%----------------------------------------------------------------------------------------
\usepackage{accents}

\renewcommand*{\dot}[1]{%
  \begingroup
  \fontfamily{cmr}\selectfont{\accentset{\mbox{\bfseries\hspace{0.16ex}.}}{#1}}
  \endgroup
}

\renewcommand*{\ddot}[1]{%
  \begingroup
  \fontfamily{cmr}\selectfont{\accentset{\mbox{\bfseries\hspace{0.16ex}.\hspace{-0.25ex}.}}{#1}}
  \endgroup
}

%----------------------------------------------------------------------------------------
% USEFUL MAKROS
%----------------------------------------------------------------------------------------

\newcommand{\N}{\mathbb{N}}
\newcommand{\Z}{\mathbb{Z}}
\newcommand{\Q}{\mathbb{Q}}
\newcommand{\R}{\mathbb{R}}
\newcommand{\C}{\mathbb{C}}

