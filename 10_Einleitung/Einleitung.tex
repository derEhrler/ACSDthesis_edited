\chapter{Einleitung} \label{chap:Einleitung}
\thispagestyle{empty}
Die modellprädiktive Regelung (MPC) hat sich in vielen Industrien, einschließlich der Automobilindustrie, als effektive Methode zur Vorhersage und Regelung dynamischer Systeme herausgestellt. Im Bereich des automatisierten Fahrens können durch die Anwendung von MPC optimale Fahrentscheidungen in Echtzeit getroffen werden, die den Komfort und die Sicherheit des Fahrzeugs erhöhen. Bezogen auf die sehr hohen Ansprüche an Sicherheit und im besonderen Maße an die Zuverlässigkeit solcher Systeme, sollen im Rahmen dieses Pflichtpraktikums realitätsnahe Fahrszenarien entwickelt werden, welche mittels festgelegter Key Performance Indicators (KPIs) in einer Test-Pipeline bewertet werden sollen.\\

\noindent\textbf{Aufgabenstellung}\\
\noindent Im Rahmen des Praktikums soll eine automatisierte Teststrategie für dieses Modul konzeptioniert und implementiert werden.Teilaufgaben dabei sind:
\begin{itemize}
    \item Konzeptentwicklung
    \item Definition geeigneter Fahrszenarien
    \item Parametrierung
    \item Definitionvon KPIs zur Beurteilung der Leistung der MPC
    \item Einbindung in eine Gitlab CI Pipeline
\end{itemize}

\noindent\textbf{Strukturierung}\\
\noindent Im nächsten Kapitel wird zunächst eine Einführung in alle Themenbereiche gegeben. Insbesondere wird der modellprädiktive Pfadfolgeregler näher erläutert, um ein besseres Systemverständnis zu erlangen und so die Qualität der Tests zu steigern. Außerdem wird auf  Strategien für das Überprüfen von Software und Testabläufe, speziell in der Automobilindustrie, näher eingegangen und ein Einblick in aktuelle Softwarentwicklungsabläufe mit \textit{Continious Integration (CI)} und \textit{Continious Deployment (CD)} gegeben.

Anschließend werden die implementierten Tests vorgestellt. Dies beinhaltet den Aufbau der Testklassen, die Parametrierung der Fahrszenarien und die Definition von Kriterien für die Bewertung der Simulationsergebnisse, \textit{Key Performance Indicators (KPI)}.

Abschließend wird der Ablauf in der Gitlab CI Pipeline und einige durch die Tests aufgedeckte Probleme der aktuellen Reglerimplementierung vorgestellt.
